\documentclass[a4paper,10pt]{article}
\usepackage[hmargin=1.3cm,vmargin=2cm,twocolumn]{geometry}
\usepackage[utf8x]{inputenc}
\usepackage{hyperref}
\usepackage{makeidx}
\makeindex
\usepackage{amsmath}
\usepackage{stmaryrd}
\usepackage{amssymb}

\newcommand{\slidesone}[1]{slides (\href
{http://lara.epfl.ch/w/_media/sav12:sav12-lecture01.pdf?cache=cache}
{lecture 1}, #1)}
\newcommand{\slidestwo}[1]{slides (\href
{http://lara.epfl.ch/w/_media/sav12:sav12-lecture02.pdf?cache=cache}
{lecture 2}, #1)}
\newcommand{\slidesthree}[1]{slides (\href
{http://lara.epfl.ch/w/_media/sav12:sav12-lecture03.pdf?cache=cache}
{lecture 3}, #1)}
\newcommand{\slidesfour}[1]{slides (\href
{http://lara.epfl.ch/w/_media/sav12:sav12-lecture04.pdf?cache=cache}
{lecture 4}, #1)}
\newcommand{\slidesfive}[1]{slides (\href
{http://lara.epfl.ch/w/_media/sav12:sav12-lecture05.pdf?cache=cache}
{lecture 5}, #1)}

\newcommand{\HLB}{\href{http://lara.epfl.ch/web2010/sav09:hoare_logic_basics}
{Hoare Logic Basics, week 2}}
\newcommand{\CVCG}{\href{http://lara.epfl.ch/web2010/sav09:compositional_vcg}
{Compositional VCG, weeks 2-3}}
\newcommand{\FVCG}{\href{http://lara.epfl.ch/web2010/sav08:forward_vcg}
{Forward VCG, week 3}}
\newcommand{\BVCG}{\href{http://lara.epfl.ch/web2010/sav08:backward_vcg}
{Backward VCG, week 3}}
\newcommand{\SRHL}{\href
{http://lara.epfl.ch/web2010/sav09:syntactic_rules_for_hoare_logic}
{Syntactic rules for Hoare logic, week 3}}

\newcommand{\sqleq}{\ensuremath{\sqsubseteq}}

%opening
\title{Synthesis, Analysis \& Verification\\Topics Index}
\author{Diggory Hardy}

\begin{document}

\maketitle
\begin{abstract}
 will the program crash, does it terminate, is the result correct, how long
will it take to run, how much power will it consume, will it interfere with
other jobs, will it leak private information
\end{abstract}
This is a summary of notes available at \url{http://lara.epfl.ch/w/sav12:top}.

\tableofcontents

\section{Introduction}
\subsection{Verification Condition Generation}\index{verification condition
generation|textbf}
The first step in proving program correctness is generating verification
conditions.
Example in \slidesthree{p3}.

Section \ref{coderep} covers representation of code and predicates.

\subsubsection*{Providing a specification}\index{specification}
Intro \slidesone{p33}: add preconditions \verb$require( ... )$ on inputs
and postcondition \verb$ensuring( ... )$ on output.

\textbf{Loop invariants:}\index{loop invariant}
Intro \slidesone{p38}: a condition which holds when initially
encountered, which is preserved by the loop, and is strong enough to prove
the postcondition.

\subsection{Proving VCs}
Once you've got verification conditions, you can attempt to prove them or find
counter-examples. Section {logic} covers the logic systems needed to reason
about VCs, while section {abstrinterp} looks at how abstractions can simplify
code sufficiently to reason about non-trivial functions.



%%%%%%%%%%%%%%%%%%%%%%%%%%%%%%%%%%%%%%%%%%%%%%%%%
\label{coderep}
\section{Code representations}
\subsection{Guarded Command Language}\index{guarded command language|textbf}
\slidesthree{p17}:
\begin{itemize}
 \item \verb$assume(F)$ \index{assume} — $F$ is required to hold in any
execution (assume any exceptions didn't happen)
 \item \verb$s1 ; s2$ \index{sequential composition} — do $s_1$ then $s_2$
 \item \verb$s1 [] s2$ \index{drunk if} — do $s_1$ or $s_2$ arbitrarily (drunk
if)
 \item \verb$s*$ — execute $s$ any number of times
 \item \verb$havoc(x)$ \index{havoc} — after execution, $x$ may have any value
while other variables remain unchanged
\end{itemize}
Guarded commands can be mapped to relations: \slidesthree{18}.

$x = E$ can be written as \verb$havox(x); assume(x==E)$ (slide 25).

\textbf{Computing from a program:} \index{guarded command language!computing}
\slidesfour{p3}

Map \verb$while(C){s1}; s2$ to \verb$( assume(C); s1 )* ; assume(!C); s2$
\index{guarded command language!loops}.

\subsection{Relations} \index{relations|textbf} \index{non-determinisity}
Programs are not always deterministic (in theory, they allow randomness):
\slidesthree{p15}. We use relations: $(x,x')∈r$ if and only if the
operation specified by $r$ can map state $x$ to $x'$. Relations are not
automatically functions: for some initial state $x$ there can be zero, one or
several results $x'$.

\textbf{Identity relation:} \index{relations!identity} The identity relation on
set $S$ is $\Delta_S = \{(s,s) | s∈S\}$.

\textbf{Computing:} \index{relations!computing}
pages 2 onwards, slide set 4, week 2.

\textbf{Normal Form Theorem:}\index{normal form} \slidesfour{p13}

\subsubsection{Hoare Logic} \index{Hoare logic|textbf}\index{Hoare triple}
Basics: \HLB.
Hoare triples are some $\{P\} r \{Q\}$ where $P$ is a precondition, $r$ a
relation and $Q$ a postcondition. Usually $P,Q$ are subsets of $S$, the set of
all states.
\[ \{P\} r \{Q\} \mbox{ means } \forall s,s'∈S.(s∈P \land (s,s')∈r \rightarrow
s'∈Q \]

One can calculate some $P$ or $Q$ satisfying the triple:
the \textbf{weakest precondition} \index{weakest precondition} or
\textbf{strongest postcondition}\index{strongest postcondition}.
\begin{align*}
 wp(r,Q) &= \{s∈S | \forall s'∈S. (s,s') ∈r \rightarrow s'∈Q\} \\
 sp(P,r) &= \{s'| \exists s.s∈P \land (s,s')∈r\}
\end{align*}

Obtaining verification conditions for Hoare triples:
\CVCG (at
end).

Manipulations (strengthening, weakening, some stuff on loops and composition):
\SRHL.

\subsubsection{Operations on relations}
From \CVCG:

Assignment, assume, havoc.\index{assignment}\index{assume}\index{havoc}

Union, sequential composition \index{relations!composing} \index{sequential
composition}:
\[ \{(x,y)|f(x,y)\} \circ \{(y,z)|g(y,z)\} = \{(x,z)|\exists y. f(x,y) \land
g(y,z) \} \]

More: \FVCG, \BVCG.

\subsubsection{Invariants and loops} \index{Hoare logic!invariants}
Introduction to invariants on loops for Hoare triples:
\href{http://lara.epfl.ch/w/sav12:invariant_inference_perspective}{Monday, week
4}.

\subsection{Control Flow Graphs} \index{control flow graphs}
Transition systems are essentially relations between states: 
\href{http://lara.epfl.ch/w/sav11:transition_system}{Monday, week 4}
(also contains a note about nested loops). CFGs are graphs of states and
transitions.


%%%%%%%%%%%%%%%%%%%%%%%%%%%%%%%%%%%%%%%%%%%%%%%%%
\label{abstrinterp}
\section{Abstract Domains} \index{abstract domains}
\subsection{Introduction}
The basic idea of abstract interpretation is to consider the program as a
control flow graph on sets of states. From Monday, week 4:
\begin{itemize}
\item
\href{http://lara.epfl.ch/w/_media/sav12:lecture8.pdf?cache=cache}{slides}
\item
\href{http://lara.epfl.ch/w/cc09:sets_of_states_at_each_program_point}
{\emph{Sets of states at each program point}}
\item \href{http://lara.epfl.ch/w/cc09:range_analysis_example}
{\emph{Range analysis}} \index{abstract interpretation!range analysis}
\end{itemize}

\subsection{Lattices} \index{lattices}
Introduced Tuesday week 4
(\href{http://lara.epfl.ch/w/_media/sav12:lecture9.pdf?cache=cache}{slides}).

\subsubsection{Partial order}
\href{http://lara.epfl.ch/w/partial_order}{\textbf{Partial order}}
\index{partial order}
($≤$): must, for all $x,y,z$, satisfy
\begin{itemize}
 \item $x≤y$ (reflexivity)
 \item $x≤y \land y≤x \rightarrow x=y$ (antisymmetry)
 \item $x≤y \land y≤x \rightarrow x≤z$ (transitivity)
\end{itemize}
\emph{Upper bound}, \emph{maximal element}, \emph{greatest element},
\emph{least upper bound} (lub, $\sqcup$), etc. and \emph{monotonic
functions} are defined on the same page.

\subsubsection{Lattices}
Defn: a \textbf{lattice} is a partial order in which every two-element set has
a least upper bound and greatest lower bound.
\href{http://lara.epfl.ch/w/sav08:lattices}{Definition and corollaries}.

Lattices are used to
\href{http://lara.epfl.ch/w/sav09:partial_orders_for_approximation}
{approximate states}.

Products of lattices can themselves be lattices:
\href{http://lara.epfl.ch/w/sav09:products_of_lattices}
{Monday week 5}. \index{lattices!products of}

\subsubsection{Fixed points} \index{fixed points}
Given a function $f:A \rightarrow A$ on some domain $A$, $x∈A$ is a fixed point
of $f$ iff $f(x) = x$. See \href{http://lara.epfl.ch/w/sav08:fixpoints}{wiki}.

\textbf{Tarski's fixed point theorem}:
\href{http://lara.epfl.ch/w/sav08:tarski_s_fixpoint_theorem}{week 4}
\index{fixed point!Tarski's theorem} \index{Tarski's fixed point theorem}
More definitions:
\begin{itemize}
 \itemsep 0pt
 \item $Post = \{x|G(x) \sqleq x\}$\index{post}
 \item $Pre = \{ x | x \sqleq G(x) \}$
 \item $Fix = \{x|G(x) = x\}$
\end{itemize}
Then (Tarski's thm): Let $(A, \sqleq)$ be a complete lattice and $G:A
\rightarrow A$ be a monotonic function. Then $\bigsqcap Post$ is the least
element of $Fix$ and $\bigsqcup Pre$ is the greatest element of $Fiv$.

\subsubsection{Galois Connection}
From \href{http://lara.epfl.ch/w/sav09:galois_connection}{wiki}: a Galois
connection is defined by two monotonic functions $\alpha:C → A$ and $\gamma:A→C$
where $(C,≤)$ and $(A,\sqleq)$ are partial orders and for all $a∈A$ and $c∈C$,
\[ \alpha(c) \sqleq a \Leftrightarrow c ≤ \gamma(a) \]

Usage terminology:
\href{http://lara.epfl.ch/w/sav12:using_galois_connection_in_abstraction_interpretation}
{wiki}.

\subsection{Transition systems} \index{transition systems}
Transition systems are given by a set of initial states $Init \subseteq S$ and
a relation $r \subseteq S^2$. The set of reachable states is then
$sp(Init,r^*)$. Introduction, with notes on fixed points:
\href{http://lara.epfl.ch/w/sav12:abstract_interpretation_of_transition_system}
{Monday, week 5}.

\subsubsection{Collecting semantics} \index{collecting semantics}
Collecting semantics give us, for every program point/vertex in CFG a set of
possible variable states (i.e. a mapping $V → 2^S$ for program points $V$ and
variable states $S$). Introduction:
\href{http://lara.epfl.ch/w/sav11:transition_system_and_collecting_semantics}
{\emph{Transition systems and collecting semantics}, Tuesday week 5}.
More on this: 
\href{http://lara.epfl.ch/w/sav09:collecting_semantics_for_example_program}
{example},
\href{http://lara.epfl.ch/w/sav09:collecting_semantics}
{additional notes}.

From Monday week 6,
\href{http://lara.epfl.ch/w/sav09:variable_range_analysis_for_example_program}
{example using ranges as abstract domain}. \index{range analysis!example}

\subsubsection{Abstract reachability} \index{abstract reachability}
Abstract reachability tells us, given a set of predicates, which predicates
hold at each node of the CFG.
\href{http://lara.epfl.ch/w/_media/sav12:predabs1.pdf?cache=cache}
{Hossein's slides (Friday, week 5).}


%%%%%%%%%%%%%%%%%%%%%%%%%%%%%%%%%%%%%%%%%%%%%%%%%
\label{logic}
\section{Logic}

\subsection{Presburger Arithmetic}\index{Presburger arithmetic|textbf}
Introduced \slidestwo{p3}.

\textbf{Propositional formulas:}\index{propositional formulas} expressions
involving the constants $true, false$, propositional variables $p,q, ...$ and
logical operators $\land \lor \lnot \rightarrow \leftrightarrow$. $p \rightarrow
q$ is equivalent to $\lnot p \lor q$.

\textbf{First order logic:}\index{first order logic} extension of propositional
logic with equality $=$, predicate symbols $P,Q,...$, function symbols
$f,g,...$, variables $x,y,...$ in some domain $D$, for-all and exists
quantifiers $\forall \; \exists$.
First order logic formulas are the closure of the set:
\[ \{\lnot P, P \land Q,
P\lor Q, P \rightarrow Q, P \leftrightarrow Q, \forall x.P, \exists x.P | P,Q
\mbox{ are first-order formulas} \} \]

Formulas are \textbf{valid}\index{valid} if for all interpretations of
unqualified symbols they are true, \textbf{satisfiable}\index{satisfiable} if
for some interpretation they are true, and
\textbf{unsatisfiable}\index{unsatisfiable} if not satisfiable.

\subsubsection{Quantifier Elimination}\index{quantifier elimination}
This was covered in the \href{http://lara.epfl.ch/w/sav12:lab_03}{Friday,
week 3}, and allows rewriting Presburgher arithmetic formulas without
existential quantifiers.

\subsubsection{More}
Solving (exercise, hints at efficient code):
\href{http://lara.epfl.ch/w/sav12:calculating_solutions_of_pa_formulas}{Monday,
week 4} \index{Presburger arithmetic!solving}



\section{Loop Invariant Inference}
???

\textbf{predicate abstraction}

\textbf{abstract interpretation and data-flow analysis}

\textbf{pointer analysis, typestate}


%\clearpage
\addcontentsline{toc}{section}{Index}
\printindex

\end{document}
